\setmainfont{OpenSans}[
	Path=./,
	Extension=.ttf,
    UprightFont = *-Regular,
    BoldFont = *-SemiBold,
	ItalicFont = *-Italic,
	% SmallCapsFont = TeX Gyre Termes,
	% SmallCapsFeatures={Letters=SmallCaps}
]
\newfontfamily{\lato}{Lato}[
Path=./,
	Extension = .ttf,
	UprightFont = *-Regular,
    BoldFont = *-Bold
]
\usepackage{textcomp}
\pagestyle{empty}
\setlength{\tabcolsep}{0em}
\linespread{1.1}

% Custom definitions

\newcommand{\ezbullet}{$\boldsymbol{\cdot}$}
% Subsections
\newcommand\ruleafter[1]{#1~\xhrulefill{gray}{1pt}}

\titleformat{\subsection}
       {\normalfont\lato\fontsize{14}{16}\bfseries}
       {\thesubsection}
       {1em}{\ruleafter}[]

\setlist[itemize]{noitemsep, topsep=0pt, label=\ezbullet}


% tab
\newcommand\tab[1][1cm]{\hspace*{#1}}

% indentsection style, used for sections that aren't already in lists
% that need indentation to the level of all text in the document
\newenvironment{indentsection}[1]%
{\begin{list}{}%
	{\setlength{\leftmargin}{#1}}%
	\item[]%
}
{\end{list}}

% opposite of above; bump a section back toward the left margin
\newenvironment{unindentsection}[1]%
{\begin{list}{}%
	{\setlength{\leftmargin}{-0.5#1}}%
	\item[]%
}
{\end{list}}

% format two pieces of text, one left aligned and one right aligned
\newcommand{\headerrow}[2]
{#1 \hfill #2 \\}

\addtolength{\topmargin}{-.2in}
\addtolength{\textheight}{.4in}

\usepackage{comment}
\usepackage{etoolbox}

\newtoggle{dense}

